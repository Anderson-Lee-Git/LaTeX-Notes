\chapter{Introduction}
\lecture{1}{13 Oct. 08:00}{First Lecture}
\section{A Template for you to take note}
This is a simple demo for you to take fancy notes in \LaTeX!

\section{Useful Environment}
We now see some common environment you'll need to complete your note.

\begin{definition}[Natural number]\label{def}
	We denote the set of \emph{natural numbers} as \(\mathbb{\MakeUppercase{n}} \).
\end{definition}


\begin{lemma}[Useful lemma]\label{lma}
	Given the axioms of \hyperref[def]{natural numbers \(\mathbb{\MakeUppercase{n}} \)}, we have
	\[
		0\neq 1.
	\]
\end{lemma}
\begin{proof}[An obvious proof]
	Obvious.
\end{proof}
\begin{proposition}[Useful proposition]\label{prop}
	From \autoref{lma}, we have
	\[
		0<1.
	\]
\end{proposition}

\begin{theorem}[Mass-energy equivalence]\label{thm}
	Given \autoref{prop}, we then have
	\[
		E = mc^2.
	\]
\end{theorem}
\begin{proof}
	The blank left for me is too small,\footnote{\url{https://en.wikipedia.org/wiki/Richard_Feynman}} hence we put the proof in \autoref{appendix}.
\end{proof}

\begin{corollary}[Riemann hypothesis]\label{col}
	From \autoref{thm}, we then have the following.

	\par The real part of every nontrivial zero of the Riemann zeta function is \(\frac{1}{2}\), where the
	Riemann zeta function is just
	\[
		\zeta (s)=\sum _{n=1}^{\infty }{\frac {1}{n^{s}}}={\frac {1}{1^{s}}}+{\frac {1}{2^{s}}}+{\frac {1}{3^{s}}}+\cdots.
	\]
\end{corollary}
\begin{proof}
	The proof should be trivial, we left it to you.\todo{DIY (Do It Yourself)}
\end{proof}
\begin{prev}
	We see that \autoref{lma} is really helpful in the proof!
\end{prev}
\begin{remark}
	This leads to lots of useful theorem!
\end{remark}
\begin{note}
	I hope you learn something while proving this!
\end{note}
\begin{eg}
	Here are some applications of \autoref{col}, see the link.\footnote{\url{https://math.stackexchange.com/questions/404624/what-does-proving-the-riemann-hypothesis-accomplish}}
\end{eg}

\subsubsection{Internal Link}
You should see all the common usages of internal links. Additionally, we can use citations as \cite{newton1726philosophiae}, which just link
to the reference page!

\section{Figures}
A simple demo for drawing:
\begin{figure}[H]
	\centering
	\incfig{test}
	\caption[Caption]{Random Drawing.\footnotemark}
	\label{fig:test}
\end{figure}
\footnotetext{For detailed information, please see \url{https://github.com/sleepymalc/VSCode-LaTeX-Inkscape}.}
