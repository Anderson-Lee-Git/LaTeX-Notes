\usepackage{amsmath, amsfonts, mathtools, amsthm, amssymb}
\usepackage{geometry}
\usepackage{float}
\usepackage{mathrsfs}
\usepackage{booktabs}
\usepackage{mathtools}
\usepackage{enumerate}
\usepackage{graphicx}
\usepackage{indentfirst}
\usepackage{xcolor}
\usepackage{color}
\usepackage{bm}
\usepackage{caption}
\usepackage{array}
\usepackage{pdfpages}
\usepackage{etoolbox}
\usepackage{physics}
\usepackage{siunitx}
\usepackage[natbib=true,style=numeric,sorting=none]{biblatex}
\usepackage[T1]{fontenc}
\usepackage{inconsolata}
\usepackage[nodayofweek]{datetime}
\usepackage[useregional]{datetime2}
\usepackage{import}
\usepackage{transparent}
\usepackage{lastpage}
\usepackage[colorlinks=true,linkcolor=blue,urlcolor=red]{hyperref}
\usepackage{tikz-cd}
% quiver style
\usepackage{tikz-cd}
% `calc` is necessary to draw curved arrows.
\usetikzlibrary{calc}
% `pathmorphing` is necessary to draw squiggly arrows.
\usetikzlibrary{decorations.pathmorphing}

% A TikZ style for curved arrows of a fixed height, due to AndréC.
\tikzset{curve/.style={settings={#1},to path={(\tikztostart)
					.. controls ($(\tikztostart)!\pv{pos}!(\tikztotarget)!\pv{height}!270:(\tikztotarget)$)
					and ($(\tikztostart)!1-\pv{pos}!(\tikztotarget)!\pv{height}!270:(\tikztotarget)$)
					.. (\tikztotarget)\tikztonodes}},
	settings/.code={\tikzset{quiver/.cd,#1}
			\def\pv##1{\pgfkeysvalueof{/tikz/quiver/##1}}},
	quiver/.cd,pos/.initial=0.35,height/.initial=0}

% TikZ arrowhead/tail styles.
\tikzset{tail reversed/.code={\pgfsetarrowsstart{tikzcd to}}}
\tikzset{2tail/.code={\pgfsetarrowsstart{Implies[reversed]}}}
\tikzset{2tail reversed/.code={\pgfsetarrowsstart{Implies}}}
% TikZ arrow styles.
\tikzset{no body/.style={/tikz/dash pattern=on 0 off 1mm}}

% useful macro for class
\newcommand{\probability}[2]{\mathbb{\MakeUppercase{P}}_{#1} \left(#2\right)}
\newcommand{\variance}[2]{\mathrm{Var}_{#1} \left[ #2 \right]}
\newcommand{\expectation}[2]{\mathbb{\MakeUppercase{E}}_{#1} \left[#2\right]}
\newcommand{\at}[3]{\left.#1\right\vert_{#2}^{#3}}
\newcommand\quotient[2]{
	\mathchoice
	{% \displaystyle
		\text{\raise1ex\hbox{$#1$}\Big/\lower1ex\hbox{$#2$}}%
	}
	{% \textstyle
		#1\,/\,#2
	}
	{% \scriptstyle
		#1\,/\,#2
	}
	{% \scriptscriptstyle  
		#1\,/\,#2
	}
}
\newcommand{\identity}{\mathrm{id}}
\newcommand{\sinc}{\mathop{\mathrm{sinc}}}
\newcommand{\rect}{\mathop{\mathrm{rect}}}
\newcommand{\tri}{\mathop{\mathrm{tri}}}
\newcommand{\Real}{\mathop{\mathrm{Re}}}

\newcommand{\Homomorphism}{\mathrm{Hom}}
\newcommand{\Morphism}{\mathrm{Mor}}
\newcommand{\Object}{\mathrm{Ob}}

\DeclareMathOperator{\im}{Im}
\DeclareMathOperator{\sgn}{sgn}
\DeclareMathOperator{\trace}{tr}
\DeclareMathOperator{\rank}{rank}

\usepackage{bbm}
% draw the basic plot
%\usepackage{pgfplots}
%\pgfplotsset{compat=1.17}

% when you need multi-row in tabular
\usepackage{multirow}

\geometry{a4paper,left=2.54cm,right=2.54cm,top=2.54cm,bottom=2.54cm}
\setlength{\headheight}{13.59999pt}
\setlength{\parindent}{1em}
\thispagestyle{empty}

\newcolumntype{P}[1]{>{\centering\arraybackslash}p{#1}}
\renewcommand{\qed}{\hfill$\blacksquare$}

\renewcommand{\emph}[1]{{\color{Turquoise3}\textsl{#1}}}
\definecolor{Turquoise3}{RGB}{0, 134, 139}

\newcommand{\incfig}[1]{%
	\def\svgwidth{\columnwidth}
	\import{./Figures/}{#1.pdf_tex}
}

\newcommand\blfootnote[1]{%
	\begingroup
	\renewcommand\thefootnote{}\footnote{#1}%
	\addtocounter{footnote}{-1}%
	\endgroup
}

% TodoNotes and inline notes in fancy boxes
\usepackage{todonotes}
\usepackage{tcolorbox}

% Add \conta symbol to denote contradiction
\usepackage{stmaryrd} % for \lightning
\newcommand\conta{\scalebox{1.5}{$\lightning$}}

% Algorithm Env
\usepackage[linesnumbered,lined,vlined,ruled,commentsnumbered]{algorithm2e}
\SetKwComment{Comment}{// }{}
\SetArgSty{textsl}

\pdfsuppresswarningpagegroup=1
\usepackage{fancyhdr} % Required for customizing headers and footers

\pagestyle{fancy} % Enable custom headers and footers

\lhead{\small\Course\ifdef{\Instructor}{\ (\Instructor):}{}\ \Title} % Left header; output the instructor in brackets if one was set
\chead{} % Centre header
\rhead{}

\rhead{\small\ifdef{\DueDate}{Due\ \DueDate}{}} % Right header; output the author name if one was set, otherwise the due date if that was set


\lfoot{} % Left footer
\cfoot{\small Page\ \thepage\ of\ \pageref{LastPage}} % Centre footer
\rfoot{} % Right footer

% Appendix environment
\usepackage{appendix}
\def\sectionautorefname{Section}
\def\subsectionautorefname{Section}
\def\appendixautorefname{Appendix}
\renewcommand\appendixname{Appendix}
\renewcommand\appendixtocname{Appendix}
\renewcommand\appendixpagename{Appendix}
% begin appendix autoref patch [\autoref subsections in appendix](https://tex.stackexchange.com/questions/149807/autoref-subsections-in-appendix)
\makeatletter
\patchcmd{\hyper@makecurrent}{%
	\ifx\Hy@param\Hy@chapterstring
		\let\Hy@param\Hy@chapapp
	\fi
}{%
	\iftoggle{inappendix}{%true-branch
		% list the names of all sectioning counters here
		\@checkappendixparam{chapter}%
		\@checkappendixparam{section}%
		\@checkappendixparam{subsection}%
		\@checkappendixparam{subsubsection}%
		\@checkappendixparam{paragraph}%
		\@checkappendixparam{subparagraph}%
	}{}%
}{}{\errmessage{failed to patch}}

\newcommand*{\@checkappendixparam}[1]{%
	\def\@checkappendixparamtmp{#1}%
	\ifx\Hy@param\@checkappendixparamtmp
		\let\Hy@param\Hy@appendixstring
	\fi
}
\makeatletter

\newtoggle{inappendix}
\togglefalse{inappendix}

\apptocmd{\appendix}{\toggletrue{inappendix}}{}{\errmessage{failed to patch}}
\apptocmd{\subappendices}{\toggletrue{inappendix}}{}{\errmessage{failed to patch}}
% end appendix autoref patch

% Environments
\makeatother
% For box around Definition, Theorem, etc.
\usepackage{mdframed}
\mdfsetup{skipabove=1em,skipbelow=0em}
\theoremstyle{definition}
\newmdtheoremenv[nobreak=true]{definition}{Definition}[section]
\providecommand*\definitionautorefname{Definition}
\newmdtheoremenv[nobreak=true]{theorem}{Theorem}[section]
\providecommand*\theoremautorefname{Theorem}
\newmdtheoremenv[nobreak=true]{lemma}{Lemma}[section]
\providecommand*\lemmaautorefname{Lemma}
\newmdtheoremenv[nobreak=true]{corollary}{Corollary}[section]
\providecommand*\corollaryautorefname{Corollary}
\newmdtheoremenv[nobreak=true]{proposition}{Proposition}[section]
\providecommand*\propositionautorefname{Proposition}

\makeatletter


\renewcommand\headrulewidth{0.5pt}

\makeatletter
\def\@maketitle{%
	\newpage
	\null
	\vskip 2em%
	\begin{center}%
		\let \footnote \thanks
		{\LARGE \@title \par}%
		\vskip 1.5em%
			{\large
				\lineskip .5em%
				\begin{tabular}[t]{c}%
					\@author \\
					\normalsize{
						\ifdef{\Uniquname}{Uniquname: \Uniquname}{}
						\hspace{1em}
						\ifdef{\UMID}{umid: \UMID}{}
					}
				\end{tabular}\par}%
		\vskip 1em%
			{\large \@date}%
	\end{center}%
	\par
	\vskip 1.5em
}
\makeatother

\title{
	\begin{center}
		\rule{15cm}{0.01cm}
		\\\LARGE{
			\Instuition
			\\
			\SeasonYear
			\\
			\Course
		}
		\\\rule{15cm}{0.01cm}
		\\\vspace{6cm}
		\begin{huge}
			\sc{\Title}
		\end{huge}
	\end{center}
	\vfill
	\flushleft
	\ifdef{\Group}{
		\begin{center}
			\large
			\sc{\Group}
			\mbox{}
		\end{center}
	}{}
}

